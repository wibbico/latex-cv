\documentclass[
    fontsize=11pt,
    paper=a4,
    parskip=half,
    fromlocation=false,
    foldmarks=false,
    numericaldate=false,
    firstfoot=false,
    backaddress=false,
    fromalign=right,
]{scrlttr2}

\usepackage[utf8]{inputenc}
\usepackage[german]{babel}
\usepackage[T1]{fontenc}
\usepackage{helvet}
\renewcommand{\familydefault}{\sfdefault}
\usepackage{graphicx}
\usepackage{xcolor}
\usepackage{hyperref}

% Geometry adjustments for DIN 5008 letter
\usepackage[margin=20mm,marginparwidth=0pt]{geometry}

% Colors
\definecolor{primary}{RGB}{0, 102, 204}

% Sender information
\setkomavar{fromname}{<VAR>anschreiben.contact.name</VAR>}
\setkomavar{fromemail}{<VAR>anschreiben.contact.email</VAR>}
\setkomavar{fromphone}{<VAR>anschreiben.contact.phone</VAR>}

% Sender address (displayed in sender block, hidden from window via backaddress=false)
\setkomavar{fromaddress}{%
<BLOCK>if anschreiben.sender_address and anschreiben.sender_address.street</BLOCK>
  <VAR>anschreiben.sender_address.street</VAR> \\
<BLOCK>endif</BLOCK>
<BLOCK>if anschreiben.sender_address and anschreiben.sender_address.postal_city</BLOCK>
  <VAR>anschreiben.sender_address.postal_city</VAR>
<BLOCK>endif</BLOCK>
}

% Recipient address (company address for letter window)
\setkomavar{toname}{<VAR>anschreiben.company_name</VAR><BLOCK>if anschreiben.contact_person</BLOCK>\\ <VAR>anschreiben.contact_person</VAR><BLOCK>endif</BLOCK>}

\setkomavar{toaddress}{%
  <VAR>anschreiben.company_street</VAR> \\
  <VAR>anschreiben.company_postal_code</VAR> <VAR>anschreiben.company_city</VAR>
}

% Letter opening date and subject
\setkomavar{date}{<VAR>anschreiben.date</VAR>}
\setkomavar{subject}{<VAR>anschreiben.subject</VAR>}

% Reference information (job reference, etc.)
\setkomavar{customer}{<BLOCK>if anschreiben.kennziffer</BLOCK>Kennziffer: <VAR>anschreiben.kennziffer</VAR><BLOCK>endif</BLOCK>}

\begin{document}

\begin{letter}{<BLOCK>if anschreiben.contact_person</BLOCK><VAR>anschreiben.contact_person</VAR> \\ <BLOCK>endif</BLOCK><VAR>anschreiben.company_name</VAR> \\ <VAR>anschreiben.company_street</VAR> \\ <VAR>anschreiben.company_postal_code</VAR> <VAR>anschreiben.company_city</VAR>}

\medskip

% Opening
\opening{<VAR>anschreiben.opening</VAR>}

% Body paragraphs
<BLOCK>for paragraph in anschreiben.body_paragraphs</BLOCK>
<VAR>paragraph</VAR>

<BLOCK>endfor</BLOCK>

% Closing (non-centered)
\raggedright
<VAR>anschreiben.closing</VAR>

\vspace{0.5cm}
<VAR>anschreiben.signature</VAR>

\vfill

% Attachments
<BLOCK>if anschreiben.attachments</BLOCK>
\ps{Anlage<BLOCK>if anschreiben.attachments|length > 1</BLOCK>n<BLOCK>endif</BLOCK>:
\begin{itemize}
<BLOCK>for attachment in anschreiben.attachments</BLOCK>
  \item <VAR>attachment</VAR>
<BLOCK>endfor</BLOCK>
\end{itemize}}
<BLOCK>endif</BLOCK>

\end{letter}

% PDF Metadata
\hypersetup{
  pdftitle={<VAR>anschreiben.pdf_title</VAR>},
  pdfauthor={<VAR>anschreiben.pdf_author</VAR>},
  pdfsubject={<VAR>anschreiben.pdf_subject</VAR>},
  pdfkeywords={<VAR>anschreiben.pdf_keywords</VAR>},
  pdfproducer={<VAR>anschreiben.pdf_generator or 'Pixcel CV'</VAR>}
}

\end{document}
